%% fichier rapport.tex
%%%%%%%%%%%%%%%%

\documentclass[a4paper, 12pt, twoside]{report}

\usepackage[utf8]{inputenc}
\usepackage[T1]{fontenc}
\usepackage[francais]{babel}
\usepackage{eurosym}
\usepackage{graphicx}
\usepackage{wrapfig}
\usepackage[bookmarks=true, bookmarksnumbered=true, linkcolor={0 0 1}, linkbordercolor={1 1 1}, pdfborderstyle={/S/U/W 1}]{hyperref}
\setcounter{tocdepth}{5}

\makeatletter
\renewcommand{\thesection}{\@arabic\c@section}
\makeatother

\usepackage{fancyhdr}
\usepackage{amsthm}
\usepackage{amsmath}
\usepackage{amsfonts}
\usepackage{amssymb}
\usepackage{mathrsfs}

\usepackage{enumerate}

\usepackage{color}
\definecolor{mygreen}{rgb}{0,0.6,0}
\definecolor{mygray}{rgb}{0.5,0.5,0.5}
\definecolor{mymauve}{rgb}{0.58,0,0.82}

\usepackage{listings}
\lstset{ %
  backgroundcolor=\color{white},   % choose the background color; you must add \usepackage{color} or \usepackage{xcolor}
  basicstyle=\footnotesize,        % the size of the fonts that are used for the code
  breakatwhitespace=false,         % sets if automatic breaks should only happen at whitespace
  breaklines=true,                 % sets automatic line breaking
  captionpos=b,                    % sets the caption-position to bottom
  commentstyle=\color{mygreen},    % comment style
  extendedchars=true,              % lets you use non-ASCII characters; for 8-bits encodings only, does not work with UTF-8
  frame=single,                    % adds a frame around the code
  keepspaces=true,                 % keeps spaces in text, useful for keeping indentation of code (possibly needs columns=flexible)
  keywordstyle=\color{blue},       % keyword style
  language=SQL,                 % the language of the code
  numbers=left,                    % where to put the line-numbers; possible values are (none, left, right)
  numbersep=5pt,                   % how far the line-numbers are from the code
  numberstyle=\tiny\color{mygray}, % the style that is used for the line-numbers
  rulecolor=\color{black},         % if not set, the frame-color may be changed on line-breaks within not-black text (e.g. comments (green here))
  showspaces=false,                % show spaces everywhere adding particular underscores; it overrides 'showstringspaces'
  showstringspaces=false,          % underline spaces within strings only
  showtabs=false,                  % show tabs within strings adding particular underscores
  stepnumber=1,                    % the step between two line-numbers. If it's 1, each line will be numbered
  stringstyle=\color{mymauve},     % string literal style
  tabsize=2,                       % sets default tabsize to 2 spaces
  title=\lstname                   % show the filename of files included with \lstinputlisting; also try caption instead of title
}


\usepackage[top=3cm, bottom=3cm, left=3cm, right=3cm]{geometry}

\lhead{Nicolas ASSOUAD}
\rhead{Clément PASCUTTO}
\lfoot{Ecole Normale Supérieure}
\rfoot{Web Data Management}
\renewcommand{\headrulewidth}{0.4pt}
\renewcommand{\footrulewidth}{0.4pt}

\newcommand{\HRule}{\rule{\linewidth}{0.5mm}}
\pagestyle{fancy}

\newtheorem{lem1}{Lemme}

\begin{document}
%%%%%%%%%%%%%%%%

\begin{titlepage}
\begin{center}

% Upper part of the page. The '~' is needed because \\
% only works if a paragraph has started.
\includegraphics[width=0.35\textwidth]{./ENS_Logo.png}~\\[1cm]

\textsc{\Large Web Data Management}\\[0.5cm]

% Title
\HRule \\[0.4cm{ \huge \bfseries Langage Naturel Pour Thymeflow\\[0.4cm] }]

\HRule \\[1.5cm]
\end{center}

% Author and supervisor
\begin{minipage}{0.4\textwidth}
\begin{flushleft} \large
\emph{Auteurs :}\\
Nicolas ASSOUAD\\
Clément PASCUTTO\\
\end{flushleft}
\end{minipage}

\begin{center}
\vfill
% Bottom of the page
{\large \today}

\end{center}
\end{titlepage}

\newpage~

Thymeflow est une plateforme de centralisation de données. Il est possible d'intéragir avec Thymeflow à la manière d'une base de donnée. Cela est fait en utilisant le langage de requête SPARQL. Ce langage de requête est un choix cohérent car il est considéré comme un standard. Ce langage de requête permet d'intéragir avec des données en format RDF. Il peut être relativement compliqué à prendre en main, notamment pour un non-informaticien. Le but de notre projet a été de rendre le système de requête de Thymeflow plus accessible aux profanes.\\

Nous avons implémenté un compilateur du langage naturel vers le langage de requête SPARQL. Nous nous sommes appuyé sur la langue anglaise. A proprement parlé, nous ne faisons pas un traitement complet des langages naturels, nous traitons un sous ensemble de l'anglais très encadré. On parle de langage controlé.\\

Notre compilateur est basé sur les travaux de recherche de Sébastien Ferré (Irisa), notamment le compilateur SQUALL []. Le sous ensemble de la langue anglaise est interprété à l'aide d'une grammaire de Montague. L'idée est d'interprété le langage naturel comme un langage formel. Les différentes règles de la grammaire permettent de générer des lambda termes qui traduisent la sémantique de la phrase exprimée en langage naturel.\\

Le lambda terme est ensuite réduit. Il contient la sémantique de la requête initiale, il peut alors être traduit dans un langage de requête, notamment SPARQL.\\

Par exemple, la phrase "what is an author of every Book" une fois interprétée en lambda calcul puis compilée en requête SPARQL, donne la requête suivante :\\

\begin{lstlisting}
SELECT DISTINCT ?__x__59 WHERE {
  ?__x__59 rdf:type rdfs:Resource.
  FILTER NOT EXISTS {
    ?__x__60 rdf:type http://schema.org/Book.
    FILTER NOT EXISTS {
      ?__x__60 http://schema.org/author ?__x__59.
    }
  }
}
\end{lstlisting}

Dans notre version du compilateur, nous avons importé les propriétés et les types de schema.org. Il serait néanmoins facile dans intégrer d'autre.\\

Lorsque l'on enrichit la grammaire avec notamment des variables ou bien une plus forte intégration de l'algèbre relationnel (des connecteurs logique dans toutes les constructions du langage), la grammaire n'est plus LR(1). Il serait donc nécessaire de la retravailler pour pouvoir intégrer ces nouvelles fonctionalités.\\

Pour étendre les possibilités du langage, il pourrait être judicieux implémenté un système de synonimes pour les propriétés et les types importés. Cela est intégrable facilement.


\end{document}
