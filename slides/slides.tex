\documentclass{beamer}

\usepackage[utf8]{inputenc}
\usepackage{listings}

\usetheme{Warsaw}

\author{Nicolas Assouad\and Cl\'ement Pascutto}
\title[Controlled Language for Thymeflow]{Controlled Language Processing for Thymeflow\\A compilation of SQUALL to SPARQL}
\date{\today}

\begin{document}

\defverbatim[colored]\lstI{
\begin{lstlisting}[language=SQL, keywordstyle=\color{red}, basicstyle=\footnotesize]
SELECT DISTINCT ?__x__59 WHERE {
  ?__x__59 rdf:type rdfs:Resource.
  FILTER NOT EXISTS {
    ?__x__60 rdf:type http://schema.org/Book.
    FILTER NOT EXISTS {
      ?__x__60 http://schema.org/author ?__x__59.
    }
  }
}
\end{lstlisting}
}

  \begin{frame}
	   \titlepage
  \end{frame}

%%%%%%%%%%%%%%%%%%%%%%%%%%%%%%%%%%%%%%%%%%%%%%%%%%%%%%%%%%%%%%%%%%%%%%%%%%%%%%%
	\begin{frame}
		\frametitle{Thymeflow and SPARQL}
    Thymeflow is a Knowledge Base meant using the RDF specification for its data.
    ~\\
    These graphs are meant to be used with the SPARQL query language.
    \begin{itemize}
      \item [Exemples SPARQL]
    \end{itemize}
    ~\\
    SPARQL can be complex and is not appropriate in the case of non-expert users, hence requires a user interface.
	\end{frame}
%%%%%%%%%%%%%%%%%%%%%%%%%%%%%%%%%%%%%%%%%%%%%%%%%%%%%%%%%%%%%%%%%%%%%%%%%%%%%%%
  \begin{frame}
    \frametitle{A controlled query language: SQUALL}
    Based on Squall project by Sebastien Ferré (Inria Rennes).

    Goal:
    \begin{itemize}
      \item Have a simpler query language
      \item This language has to be (at least nearly) as expressive as SPARQL
    \end{itemize}
    ~\\
    SQUALL is a candidate:
    \begin{itemize}
      \item How many Person are an author of the book whose title is "Web Data" ?
      \item What is a Book ?
    \end{itemize}
  \end{frame}
%%%%%%%%%%%%%%%%%%%%%%%%%%%%%%%%%%%%%%%%%%%%%%%%%%%%%%%%%%%%%%%%%%%%%%%%%%%%%%%
  \begin{frame}
    \frametitle{The compilation process}
    A sentence is compiled to a lambda term enriched with
    some constructions from SQUALL or SPARQL:
    \begin{itemize}
      \item query construction : \texttt{whose}, \texttt{how many}, \texttt{which},...
      \item relational algebra : \texttt{and}, \texttt{or}, ...
      \item and constructions from SPARQL: \texttt{triple}, \texttt{ask},...
    \end{itemize}

    The lambda term is then translated to a SPARQL query.

  \end{frame}
%%%%%%%%%%%%%%%%%%%%%%%%%%%%%%%%%%%%%%%%%%%%%%%%%%%%%%%%%%%%%%%%%%%%%%%%%%%%%%%
  \begin{frame}
    \frametitle{Results}
    The query "what is an author of every Book" is compiled to :\\
    \lstI
  \end{frame}
%%%%%%%%%%%%%%%%%%%%%%%%%%%%%%%%%%%%%%%%%%%%%%%%%%%%%%%%%%%%%%%%%%%%%%%%%%%%%%%
  \begin{frame}
    \frametitle{Conclusion}
    Some work is still to do :
    \begin{itemize}
      \item Add logical operators (\texttt{and}, \texttt{or}, \texttt{not})
      \item Add variables (\texttt{?A}) to gain expressiveness
      \item Add more synonyms to get closer to natural language
    \end{itemize}

  \end{frame}
%%%%%%%%%%%%%%%%%%%%%%%%%%%%%%%%%%%%%%%%%%%%%%%%%%%%%%%%%%%%%%%%%%%%%%%%%%%%%%%
\begin{frame}
  \centering
  Questions ?
\end{frame}
%%%%%%%%%%%%%%%%%%%%%%%%%%%%%%%%%%%%%%%%%%%%%%%%%%%%%%%%%%%%%%%%%%%%%%%%%%%%%%%


\end{document}
